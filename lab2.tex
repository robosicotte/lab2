%Standard formatting for latex lab reports
\documentclass[leqno]{article}
\usepackage{amsmath}
\usepackage{titlesec}
\usepackage{graphicx}
\titleformat{\section}{\normalfont\bfseries\itshape}{\thesection}{0.5em}{}
\begin{document}
%Name
\begin{flushright}
Matthew Sicotte\\
Physics 326 Section 01\\
September 11, 2018
\end{flushright}
\begin{center}
	{\large \bf Lab 2: Measuring the Wavelength of Light with a Steel Ruler}
\end{center}
\section*{Introduction:}
A very common problem in physics is measuring the wavelength of light from an unknown source.  Although there are many ways to make this measurement, one of the most common methods is to use a diffraction grating, which is a series of regularly-spaced lines that either absorb or transmit light.  When light is shined through this device, it can be thought of as a superposition of coherent light waves from many secondary sources.  As a result of this, when this light is projected onto a screen, many bright fringes are visible.  By measuring the position of these bright fringes and knowing the spacing of the diffraction grating, the wavelength of the light can be determined using the formula
\begin{equation}
	d(\sin{\theta_i}-\sin{\theta_n})=n\lambda
\end{equation}
where $d$ is the spacing between lines in the grating, $\lambda$ is the wavelength of the light, $n$ is the order of the diffraction maximum, $\theta_i$ is the angle of incidence, and $\theta_n$ is the angle of the $n^{th}$ order maximum.\\

\noindent 
\end{document}
